% "{'chapitre':'slci_laplace','classe':('PSI'),'type':('td'),'titre':'Système EOS', 'source':'Concours Banque PT SIA 2016','comp':['SLCI-02','SLCI-07','SLCI-08'],'corrige':False}"

% Comp ci dessus à revoir
\setchapterimage{fig_00}


\renewcommand{\titrechapitre}{Système EOS}

\chapter*{TD \arabic{cptTD} \\ 
\titrechapitre\,-- \ifprof Corrigé \else Sujet \fi}
\addcontentsline{toc}{section}{TD \arabic{cptTD} : \titrechapitre -- \ifprof Corrigé \else Sujet \fi}

\iflivret \stepcounter{cptTD} \else
\ifprof  \stepcounter{cptTD} \else \fi \fi

\renewcommand{\leftmark}{\titrechapitre}
\renewcommand{\rightmark}{\titrechapitre}


\setcounter{question}{0}

\marginnote{\textit{Concours Banque PT SIA 2016.}}
%\marginnote{\UPSTIcompetence[2]{B2-07}}
\marginnote{\xpComp{SLCI}{02}\xpComp{SLCI}{07}\xpComp{SLCI}{08} -- À vérifier}



\begin{marginfigure}
\centering
\includegraphics[width=.9\linewidth]{fig_01}
\end{marginfigure}

EOS est un système d’imagerie qui permet l’acquisition simultanée de radiographies de face et de profil du corps entier (ou d’une zone anatomique localisée) avec une réduction de la
dose de rayons X de l’ordre de 90\% par rapport à un système radiographique conventionnel ou un scanner. Une
des originalitées du système EOS est que le patient peut prendre place dans diverses positions correspondant
aux situations de la vie courante, ce qui permet d’obtenir des images de son corps  << en charge >> et donc une
visualisation plus précise d'éventuelles pathologies (scoliose, trouble de la statique...).



\begin{marginfigure}

\includegraphics[width=\linewidth]{fig_02}
\caption{Trois chaînes de transmission de puissance en parallèle \label{cy_13_ch_01_td_03_fig_1}}
\end{marginfigure}

Le mécanisme interne, constitué d’un bras mobile, guidé par rapport au bâti par trois colonnes verticales. La motorisation est assurée par trois chaînes de transmission constituées :
\begin{itemize}
\item d’un moteur électrique à courant continu commandé par l’induit ;
\item d’un réducteur à engrenages ;
\item d’une vis à billes, afin de transformer le mouvement de rotation en mouvement de translation.
\end{itemize}
La Figure \ref{cy_13_ch_01_td_03_fig_1} détaille la composition des trois chaînes.
%, tandis que les noms et symboles des caract´eristiques des diff´erents composants seront donn´es plus loin dans le sujet dans le tableau de la Figure 13.

\subsection*{Structure de l'asservissement}
La grandeur asservie est la vitesse de translation du mobile.

La grandeur de consigne est donc une vitesse notée $v_c(t)$. Celle-ci est \textbf{adaptée} afin de fournir une tension de commande proportionnelle (de gain $A_c$) notée $u_c(t)$. Cette tension est \textbf{comparée} avec une tension $m(t)$ image de la vitess de rotation de la vis $\omega_v(t)$ mesurée par uné \textbf{génératrice tachymétrique} de gain $G_t$. L'écart entre $u_c(t)$ et $m(t)$ est noté $\varepsilon(t)$. Cet écart est corrigé par \textbf{un correcteur} qui fournit une tension d'alimentation $u_m(t)$ à \textbf{la motorisation électrique}. 
En sortie du moteur, la vitesse de rotation est transmise par un \textbf{réducteur à engrenages} de fonction de transfert $R_e(p)$ et actionne ainsi la vis de vitesse $\omega_v(t)$. La fonction de transfert de \textbf{la vis à billes} est notée $H_v(p)$. L'écrou actionne le bras mobile de vitesse $v(t)$.


\question{Cet asservissement satisfait-il les hypothèses des SLCI (Systèmes Linéaires Continus Invariants) ?}

\question{Proposer un schéma-bloc fonctionnel permettant de modéliser l'asservissement en vitesse du bras mobile.}


\subsection*{Modélisation du système par équations différentielles}

Les équations concernant le moteur sont :
$$ u_m(t) = R i(t) + e(t) 
\quad C_m(t)=K i(t) 
\quad e(t)=K \omega_m(t) 
\quad 
J \dfrac{\text{d} \omega_m(t) }{\text{d} t } = C_m(t)
$$

\question{Après avoir transformer les équations dans le domaine de Laplace, donner la fonction de transfert du moteur $H_m(p) = \dfrac{\Omega_m(p)}{U_m(p)}$.}

\question{Mettre $H_m(p)$ sous forme canonique. Préciser l'ordre et le classe de la fonction de transfert. Préciser également les pôles.}

Les équations concernant le reste du système sont : 
$
\varepsilon(t)=u_c(t)-m(t)
\quad u_c(t)=A_c v_c(t)
\quad \dfrac{\text{d}  u_m(t)}{\text{d} t}=\indice{K}{cor}\varepsilon(t) 
\quad \omega_v(t)= k \omega_m(t)
\quad v(t)= p_v \omega_v(t)
\quad m(t)= G_t \omega_v(t)
$


On note $F(p) = \dfrac{V(p)}{V_c(p)}$
\question{Déterminer $F(p)$ sous forme canonique. Préciser l'ordre et le classe de la fonction de transfert.}

\subsection*{Evaluation analytique des performances}

\begin{marginfigure}

\includegraphics[width=\linewidth]{fig_03}
\caption{Diagramme d'exigences partiel\label{cy_13_ch_01_td_03_fig_3}}
\end{marginfigure}
\question{Donner la transformée de Laplace d'un échelon d'amlitude 1 et d'une rampe de pente 1.}

\question{Déterminer la valeur finale de $v(t)$ pour une entrée échelon. Conclure vis-à-vis du cahier des charges.}

\question{Déterminer la valeur finale de $v(t)$ pour une entrée en rampe.}

\subsection*{Evaluation graphique des performances}

Une simulation a permis d’obtenir l'evolution temporelle de la vitesse du bras en réponse à une consigne en
vitesse de $\SI{10}{cm.s^{-1}}$ et d’un couple résistant constant.

\begin{center}
\centering
\includegraphics[width=.6\linewidth]{fig_04}
\end{center}

\question{Détermnier le temps de réponse et l'écart statique. Le système satisfait-il les exigences du cahier des charges ?}
%
%
%\newpage
%
%\begin{center}%[h]
%%\centering
%\begin{tabular}{p{2.5cm}p{6cm}l}
%\hline
%\textbf{Exigence} & \textbf{Critères} & \textbf{Niveau} \\
%\hline 
%\multirow{4}{2.5cm}{Contrôler le mouvement de l'habitacle} 
%& Ecart statique pour une entrée indicielle & $0\degres$ \\
%& Ecart de trainage pour une entrée en rampe unitaire & $0\degres$ \\
%& Ecart dynamique & $<1\degres$ \\
%& Temps de réponse à 5\% & $\leq \SI{0,1}{s}$ \\
%%& Marge de phase & Comprise entre $45\degres$ et $50\degres$ \\
%& Dépassement & Inférieur à 10\% \\
%\hline
%\end{tabular}
%\captionof{table}{Exigences de l'asservissement \label{cy_13_ch_01_td_02_tab_1}}
%\end{center}
%
%
%
%\subsection*{Modélisation de la structure du système}
%
%\begin{marginfigure}
%\centering
%\includegraphics[width=.9\linewidth]{fig_03}
%\end{marginfigure}
%
%
%Le mouvement de l'habitacle est asservi en position angulaire. La consigne $\alpha_c(t)$ est générée par un calculateur est adaptée par un adaptateur électronique, dont le signal de sorti est noté $u_c(t)$. Ce signal est comparé au signal $m(t)$ provenant d'un capteur de position. En sortie du comparateur, l'écart $\varepsilon(t)$ est ajustée grâce à un correcteur. 
%
%Le correcteur, grâce à une tension de commande $u(t)$, pilote un servo-distributeur hdyraulique dont l'objectif est de moduler le débit d'huile $q(t)$ qui permet de déplacer le vérin. Ce déplacement est noté $\lambda(t)$. Ce vérin permet de modifier les longueurs du triangle $OA_1B_1$ et donc de modifier l'angle $\alpha(t)$ de  l'habitacle.
%
%\question{Proposer un schéma-bloc fonctionnel pour modéliser l'asservissement en position angulaire de l'habitacle.}
%
%\question{Cet asservissement satisfait-il les hypothèses des SLCI (Systèmes Linéaires Continus Invariants) ?}
%
%
%
%\subsection*{Modèle de connaissance}
%Les équations suivantes modélisent le comportement des composants :
%\begin{itemize}
%\item comparateur : $\varepsilon(t)=u_c(t)-m(t)$;
%\item correcteur : $u(t)=\varepsilon(t)$;
%\item servo-distributeur : $q(t)=K_S u(t)$;
%\item capteur : $m(t)=C\alpha(t)$, adapteur électronique : $u_c(t)=C\alpha_c(t)$;
%\item en faisant l'hypothèse que le fluide est incompressible : $q(t)=S\lambdap(t)$;
%\item en faisant l'hypothèse que les variation de $\alpha(t)$ sont limitées, on a $\alpha(t)=R\lambda(t)$.
%%($\lambdap(t)=\dfrac{\d\lambda(t)}{\d t}$).
%\end{itemize}
%
%\question{En faisant l'hypothèse que les conditions de Heaviside sont respectées; transformer les équations dans le domaine de Laplace.}
%
%\question{Calculer la fonction de transfert $F(p) = \dfrac{M(p)}{\varepsilon(p)}$. Mettre cette fonction sous forme canonique. Préciser son ordre, son gain statique, sa classe et ses pôles.}
%
%
%\question{Calculer la fonction de transfert $H(p) = \dfrac{\alpha(p)}{\alpha_c(p)}$. Mettre cette fonction sous forme canonique. Préciser son ordre, son gain statique, sa classe et ses pôles.}
%
%\subsection*{Evaluation analytique des performances}
%
%\question{Donner la transformée de Laplace d'un échelon d'amlitude 1 et d'une rampe de pente 1.}
%
%
%\question{Déterminer la valeur finale de $\alpha_c(t)$ pour une entrée échelon. Conclure vis-à-vis du cahier des charges.}
%
%\question{Déterminer la valeur finale de $\alpha_c(t)$ pour une entrée en rampe. Conclure vis-à-vis du cahier des charges.}
%
%\subsection*{Evaluation graphique des performances}
%
%Le correcteur a été modifié pour améliorer les performances du système.
%On sollicite le système avec l'entrée ci-contre.
%On mesure la réponse suivante.
%\begin{marginfigure}
%\includegraphics[width=\linewidth]{fig_04}
%\end{marginfigure}
%
%\begin{center}
%\centering
%\includegraphics[width=.4\linewidth]{fig_05}
%\end{center}
%
%\question{Le système satisfait-il les exigences du cahier des charges ?}

