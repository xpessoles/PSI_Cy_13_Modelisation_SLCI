% "{'chapitre':'slci_laplace','classe':('PSI'),'type':('td'),'titre':'Vehicule Clever à 3 roues', 'source':'Concours Banque PT SIA 2013','comp':['SLCI-02','SLCI-07','SLCI-08'],'corrige':False}"

% Comp ci dessus à revoir
\setchapterimage{fig_00}


\renewcommand{\titrechapitre}{Véhicule à trois roues Clever}

\chapter*{TD \arabic{cptTD} \\ 
\titrechapitre\,-- \ifprof Corrigé \else Sujet \fi}
\addcontentsline{toc}{section}{TD \arabic{cptTD} : \titrechapitre -- \ifprof Corrigé \else Sujet \fi}

\iflivret \stepcounter{cptTD} \else
\ifprof  \stepcounter{cptTD} \else \fi \fi

\renewcommand{\leftmark}{\titrechapitre}
\renewcommand{\rightmark}{\titrechapitre}


\setcounter{question}{0}

\marginnote{\textit{Concours Banque PT SIA 2013.}}
%\marginnote{\UPSTIcompetence[2]{B2-07}}
\marginnote{\xpComp{SLCI}{02}\xpComp{SLCI}{07}\xpComp{SLCI}{08} -- À vérifier}



\begin{marginfigure}
\centering
\includegraphics[width=.9\linewidth]{fig_01}
\includegraphics[width=.9\linewidth]{fig_02}
\end{marginfigure}

Le Clever est un démonstrateur technologique développé par un tissu d'industriels européens -- dont BMW, l'Institut Français du Pétrole (IFP) et de nombreux équipementiers --. Clever est la contraction de Compact Low Emission VEhiclefor uRban tRansportation (véhicule compact à faibles émissions pour le transport urbain). 


Du point de vue de l'architecture cinématique, le groupe motopropulseur est placé à l'arrière. À l'avant, l'habitacle repose sur une roue de moto et pivote par rapport au bloc arrière autour d'une liaison pilotée angulairement par le biais de deux vérins hydrauliques. L'inclinaison est contrôlée par un ordinateur de bord en fonction de l'angle au volant et de la vitesse. 

Le cahier des charges de l'asservissement angulaire de l'inclinaison de l'habitacle est donné par la table \ref{cy_13_ch_01_td_02_tab_1}.

\begin{center}%[h]
%\centering
\begin{tabular}{p{2.5cm}p{6cm}l}
\hline
\textbf{Exigence} & \textbf{Critères} & \textbf{Niveau} \\
\hline 
\multirow{4}{2.5cm}{Contrôler le mouvement de l'habitacle} 
& Ecart statique pour une entrée indicielle & $0\degres$ \\
& Ecart de trainage pour une entrée en rampe unitaire & $0\degres$ \\
& Ecart dynamique & $<1\degres$ \\
& Temps de réponse à 5\% & $\leq \SI{0,1}{s}$ \\
%& Marge de phase & Comprise entre $45\degres$ et $50\degres$ \\
& Dépassement & Inférieur à 10\% \\
\hline
\end{tabular}
\captionof{table}{Exigences de l'asservissement \label{cy_13_ch_01_td_02_tab_1}}
\end{center}



\subsection*{Modélisation de la structure du système}

\begin{marginfigure}
\centering
\includegraphics[width=.9\linewidth]{fig_03}
\end{marginfigure}


Le mouvement de l'habitacle est asservi en position angulaire. La consigne $\alpha_c(t)$ est générée par un calculateur est adaptée par un adaptateur électronique, dont le signal de sorti est noté $u_c(t)$. Ce signal est comparé au signal $m(t)$ provenant d'un capteur de position. En sortie du comparateur, l'écart $\varepsilon(t)$ est ajustée grâce à un correcteur. 

Le correcteur, grâce à une tension de commande $u(t)$, pilote un servo-distributeur hdyraulique dont l'objectif est de moduler le débit d'huile $q(t)$ qui permet de déplacer le vérin. Ce déplacement est noté $\lambda(t)$. Ce vérin permet de modifier les longueurs du triangle $OA_1B_1$ et donc de modifier l'angle $\alpha(t)$ de  l'habitacle.

\question{Proposer un schéma-bloc fonctionnel pour modéliser l'asservissement en position angulaire de l'habitacle.}

\question{Cet asservissement satisfait-il les hypothèses des SLCI (Systèmes Linéaires Continus Invariants) ?}



\subsection*{Modèle de connaissance}
Les équations suivantes modélisent le comportement des composants :
\begin{itemize}
\item comparateur : $\varepsilon(t)=u_c(t)-m(t)$, correcteur : $u(t)=\varepsilon(t)$;
\item servo-distributeur : $q(t)=K_S u(t)$;
\item capteur : $m(t)=C\alpha(t)$, adapteur électronique : $u_c(t)=C\alpha_c(t)$;
\item en faisant l'hypothèse que le fluide est incompressible : $q(t)=S\lambdap(t)$;
\item en faisant l'hypothèse que les variation de $\alpha(t)$ sont limitées, on a $\alpha(t)=R\lambda(t)$.
%($\lambdap(t)=\dfrac{\d\lambda(t)}{\d t}$).
\end{itemize}

\question{En faisant l'hypothèse que les conditions de Heaviside sont respectées; transformer les équations dans le domaine de Laplace.}

\question{Calculer la fonction de transfert $F(p) = \dfrac{M(p)}{\varepsilon(p)}$. Mettre cette fonction sous forme canonique. Préciser son ordre, son gain statique, sa classe et ses pôles.}


\question{Calculer la fonction de transfert $H(p) = \dfrac{\alpha(p)}{\alpha_c(p)}$. Mettre cette fonction sous forme canonique. Préciser son ordre, son gain statique, sa classe et ses pôles.}

\subsection*{Evaluation analytique des performances}

\question{Donner la transformée de Laplace d'un échelon d'amlitude 1 et d'une rampe de pente 1.}


\question{Déterminer la valeur finale de $\alpha_c(t)$ pour une entrée échelon. Conclure vis-à-vis du cahier des charges.}

\question{Déterminer la valeur finale de $\alpha_c(t)$ pour une entrée en rampe. Conclure vis-à-vis du cahier des charges.}

\subsection*{Evaluation graphique des performances}

Le correcteur a été modifié pour améliorer les performances du système.
On sollicite le système avec l'entrée ci-contre.
On mesure la réponse suivante.
\begin{marginfigure}
\includegraphics[width=\linewidth]{fig_04}
\end{marginfigure}

\begin{center}
\centering
\includegraphics[width=.5\linewidth]{fig_05}
\end{center}

\question{Le système satisfait-il les exigences du cahier des charges ?}

