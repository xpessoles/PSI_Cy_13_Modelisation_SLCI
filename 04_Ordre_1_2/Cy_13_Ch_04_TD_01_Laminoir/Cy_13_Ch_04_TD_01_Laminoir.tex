% "{'chapitre':'slci_ordre_1_2','classe':('PCSI','PTSI'),'type':('td'),'titre':'Laminoir Sendzimir', 'source':'JP Pupier','comp':['B2-07','SLCI-02','SLCI-07','SLCI-08'],'corrige':False}"
\setchapterimage{fig_00}

\renewcommand{\titrechapitre}{Laminoir Sendzimir}

\chapter*{TD \arabic{cptTD} \\ 
\titrechapitre\,-- \ifprof Corrigé \else Sujet \fi}
\addcontentsline{toc}{section}{TD \arabic{cptTD} : \titrechapitre -- \ifprof Corrigé \else Sujet \fi}

\iflivret \stepcounter{cptTD} \else
\ifprof  \stepcounter{cptTD} \else \fi \fi

\renewcommand{\leftmark}{\titrechapitre}
\renewcommand{\rightmark}{\titrechapitre}


\setcounter{question}{0}

\marginnote{\textit{Ressources JP Pupier.}}
%\marginnote{\UPSTIcompetence[2]{B2-07}}
\marginnote{XXXX\xpComp{SLCI}{02}\xpComp{SLCI}{03}\xpComp{SLCI}{07}\xpComp{SLCI}{08}}





\ifprof
\else
\begin{marginfigure}
\centering
%\includegraphics[width=3cm]{Cy_01_Ch_02_03_TD_04_qr}
\end{marginfigure}
\fi

\ifcolle
\else
\ifprof
\else
%\marginnote{
\begin{solution}
\begin{enumerate}[wide, labelwidth=!, labelindent=0pt]
\item .
\item $K_a = K_c$.
\item .
\item $K= \dfrac{1}{k_e}$, $\omega_0 =\sqrt{ \dfrac{k_ck_e}{JL}}$ et
$\xi =  \dfrac{R\sqrt{J}}{2 \sqrt{L k_ck_e}} $.
\item La courbe 2 a la plus grande valeur de $K_p$.
\item $K_i\neq 0 $ pour la courbe 3 uniquemement.
\item .
\end{enumerate}
\end{solution}
%}
\fi
\fi




