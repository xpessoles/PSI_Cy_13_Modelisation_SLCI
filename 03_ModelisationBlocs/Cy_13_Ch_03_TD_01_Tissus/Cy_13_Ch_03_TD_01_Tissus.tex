% "{'chapitre':'slci_laplace','classe':('PSI'),'type':('td'),'titre':'Tête de découpe de tissus', 'source':'Concours CCINP MP 2018','comp':['B2-07','SLCI-02','SLCI-07','SLCI-08'],'corrige':True}"
\setchapterimage{fig_00}

\renewcommand{\titrechapitre}{Tête de découpe de tissus}

\chapter*{TD \arabic{cptTD} \\ 
\titrechapitre\,-- \ifprof Corrigé \else Sujet \fi}
\addcontentsline{toc}{section}{TD \arabic{cptTD} : \titrechapitre -- \ifprof Corrigé \else Sujet \fi}

\iflivret \stepcounter{cptTD} \else
\ifprof  \stepcounter{cptTD} \else \fi \fi

\renewcommand{\leftmark}{\titrechapitre}
\renewcommand{\rightmark}{\titrechapitre}


\setcounter{question}{0}

\marginnote{\textit{Concours CCINP MP 2018.}}
%\marginnote{\UPSTIcompetence[2]{B2-07}}
\marginnote{\xpComp{SLCI}{02}\xpComp{SLCI}{03}\xpComp{SLCI}{07}\xpComp{SLCI}{08}}

\begin{marginfigure}
\centering
\includegraphics[width=.9\linewidth]{fig_01}
\end{marginfigure}

\ifprof
\else
Un système de découpe automatisé de tissus est composé (figure \ref{cy_01_ch_02_03_td_03_fig_02}):
\begin{itemize}
\item d’une table de découpe sur laquelle le tissus à découper (appelé matelas) est maintenu en position
par aspiration;
\item d’un bras transversal qui se déplace en translation de direction $\vect{y_0}$ par rapport à la table;
\item d’une tête de coupe qui se déplace en translation de direction $\vect{x_0}$ par rapport au bras transversal;
\item d’un ordinateur qui pilote l’ensemble du système.
\end{itemize}


\begin{marginfigure}%[!h]
\centering
\includegraphics[width=\linewidth]{fig_02}
\caption{Structure d’une table de découpe de tissus}
\label{cy_01_ch_02_03_td_03_fig_02}
\end{marginfigure}

\fi

\begin{marginfigure}
\centering
\includegraphics[width=\linewidth]{fig_04}
\caption{Exigence 1.2.2.1}
\label{cy_01_ch_02_03_td_03_fig_04}
\end{marginfigure}


\subsection*{Modélisation du comportement du moteur de coupe}

\begin{obj} 
Modéliser la chaîne d’asservissement en vitesse du moteur afin de déterminer les paramètres du correcteur permettant de respecter l’exigence 1.2.2.1 (figure \ref{cy_01_ch_02_03_td_03_fig_04}).
\end{obj}

\ifprof
\else
Le mouvement de coupe est asservi en vitesse. La vitesse de rotation du moteur, notée $\omega_m(t)$, est
le paramètre asservi. Elle est mesurée à l’aide d’un codeur incrémental et de son conditionneur qui
fournissent une tension $\indice{u}{mes}(t)$, image de la vitesse de rotation du moteur. Cette tension est comparée à
la tension consigne $\indice{u}{cons}(t)$, image de la vitesse de rotation de consigne $\indice{\omega}{cons}(t)$; un adaptateur fournit $\indice{u}{cons}(t)$ à partir de $\indice{\omega}{cons}(t)$. La tension écart $\varepsilon(t) = \indice{u}{cons}(t) - \indice{u}{mes}(t)$ est alors transformée en tension
d’alimentation du moteur $\indice{u}{m}(t)$ par l’ensemble correcteur-variateur.
\fi


\begin{question} \label{cy_13_ch_03_td_01_q01}
Compléter le schéma-blocs fonctionnel en indiquant dans les blocs
le nom des composants (moteur, adaptateur, correcteur-variateur, capteur-conditionneur) et les
paramètres qui transitent entre les blocs.
\end{question}
\ifprof
\begin{corrige}
\footnotesize
%\begin{figure}[!h]
\begin{center}
\begin{tikzpicture}
\sbEntree{E}

\sbBloc[5]{b0}{Adapt.}{E}
    \sbRelier[$\omega_{\text{cons}}(t)$]{E}{b0}

\sbComp{c1}{b0}
    \sbRelier{b0}{c1}

\sbBloc[3]{b1}{Correct. -- variateur}{c1}
    \sbRelier[$\varepsilon(t)$]{c1}{b1}
    
\sbBloc[3]{b2}{Moteur}{b1}
    \sbRelier[$u_m(t)$]{b1}{b2}
    

\sbSortie[4]{S}{b2}
    \sbRelier{b2}{S}
    \sbNomLien[0.8]{S}{$\omega_m(t)$}
  
%\sbRenvoi{b2-S}{c1}{}


%% BLOC DE RETOUR ET RENVOI
\sbDecaleNoeudy[4]{b2}{n1} 
\sbDecaleNoeudx[0]{n1}{n2} 

\sbBlocr{r}{Capteur}{n2} 
\sbRelieryx{b2-S}{r}
\sbRelierxy[$\indice{u}{mes}(t)$]{r}{c1}
%%%%

\end{tikzpicture}
\end{center}
\normalsize
%%%%%%%%%%%%%%%
\end{corrige}
\else
%%%%%%%%%%%%%%%%%%%%%
\footnotesize
\begin{figure}[!h]
\begin{center}
\begin{tikzpicture}
\sbEntree{E}

\sbBloc[5]{b0}{ }{E}
    \sbRelier[$\omega_{\text{cons}}(t)$]{E}{b0}

\sbComp{c1}{b0}
    \sbRelier{b0}{c1}

\sbBloc[3]{b1}{}{c1}
    \sbRelier{c1}{b1}
    
\sbBloc[3]{b2}{ }{b1}
    \sbRelier[$ $]{b1}{b2}
    

\sbSortie[4]{S}{b2}
    \sbRelier{b2}{S}
    \sbNomLien[0.8]{S}{$\omega_m(t)$}
  
%\sbRenvoi{b2-S}{c1}{}


%% BLOC DE RETOUR ET RENVOI
\sbDecaleNoeudy[4]{b2}{n1} 
\sbDecaleNoeudx[0]{n1}{n2} 

\sbBlocr{r}{}{n2} 
\sbRelieryx{b2-S}{r}
\sbRelierxy{r}{c1}
%%%%

\end{tikzpicture}
\end{center}

\end{figure}
\normalsize
%%%%%%%%%%%%%%%
\fi

\question{Le système est-il asservi ? Pourquoi ? Le système répond-il aux hypothèses des SLCI (Systèmes Linéaires Continus Invariants) ?}
\ifprof
\begin{corrige}
\begin{itemize}
\item Le système est asservi car il est équipé d'un capteur et d'un sommateur qui permettent de corriger l'écart entre la consigne de l'utilisateur et le signal mesuré sur le système.
\item Linéarité du système : à ce stade, peu d'informations permettent de répondre à la question.
\item Continuité du système : le codeur incrémental fournit une information numérique qui n'est pas continue. Sous certaines conditions, l'hypothèse de continuité pourrait être remise en question.
\item Invariance du système : on fait l'hypothèse que les constantes ne varient pas au cours du temps.
\end{itemize}
\end{corrige}
\else
\fi

\begin{question}
On note $K_a$ le gain de l'adaptateur et $K_c$ le gain du capteur. Quelle doit être la relation entre 
$K_a$ et $K_c$ pour que l'écart soit nul lorsque la vitesse du moteur est égale à la vitesse de consigne ?
\end{question}
\ifprof
\begin{corrige}
On a $\varepsilon(t) = K_a \omega_{\text{cons}}(t) - K_c \omega_m(t)$. 

Pour que $\varepsilon(t)$ soit nul lorsque $\omega_{\text{cons}}(t) =  \omega_m(t)$, il faut que 
$K_a = K_c$.

\end{corrige}
\else
\fi

\ifprof
\else
\subsubsection*{Modélisation du moteur à courant continu}

On donne les quatre équations du modèle d’un moteur à courant continu :
$u_m(t) = Ri(t) + L \dfrac{\dd i(t)}{\dd t} + e(t)$, 
$J \dfrac{\dd \omega_m(t)}{\dd t} = c_m(t) + c_r(t)$, 
$c_m(t) = k_c i(t)$, $e(t) = k_e\omega_m(t)$.
La fonction de transfert du moteur est notée $H_m(p)=\dfrac{\Omega_m(p)}{U_m(p)}$.
\fi

\ifprof
\else
\marginnote{
Le moteur utilisé est un moteur à courant continu dont les caractéristiques et les grandeurs physique sont sont :
\begin{itemize}
\item $R$, résistance de l’induit;
\item $L$, inductance de l’induit;
\item $k_e$, constante de vitesse;
\item $k_c$, constante de couple;
\item $u_m(t)$ est la tension d’alimentation du moteur;
\item $i(t)$ est l’intensité traversant l’induit;
\item $e(t)$ est la force contre-électromotrice;
\item $\omega_m(t)$ est la vitesse de rotation de l’arbre moteur;
\item $c_m(t)$ est le couple moteur;
\item $c_r(t)$ est le couple résistant;
\item $J$ est le moment d’inertie de l’ensemble en mouvement ramené à l’arbre moteur, supposé
constant dans cette partie.
\end{itemize}}
\fi

\marginnote{On notera $U_m(p) = \mathcal{L}\left(u_m(t)\right)$, 
$I(p) = \mathcal{L}\left(i(t)\right)$, 
$\Omega_m(p) = \mathcal{L}\left(\omega_m(t)\right)$, 
$C_m(p) = \mathcal{L}\left(c_m(t)\right)$, 
$C_r(p) = \mathcal{L}\left(c_r(t)\right)$, 
$E(p) = \mathcal{L}\left(e(t)\right)$.}
\begin{question}
Transformer les quatre équations dans le domaine de Laplace en supposant les conditions
initiales nulles. 
\end{question}
\ifprof
\begin{corrige}
On a $U_m(p) = RI(p) + L pI(p) + E(p)$, 
$J p\Omega_m(p) = C_m(p) + C_r(p)$, 
$C_m(p) = k_c I(p)$, 
$E(p) = k_e\Omega_m(p)$.
\end{corrige}
\else
\fi


\question{Réaliser le schéma bloc du moteur. Le moteur à courant continu seul est-il asservi ? Pourquoi ?}
\marginnote{\textit{Indications : il y a 3 blocs dans la chaîne directe et un seul dans la chaîne de retour; l'entrée du schéma est $U_m(p)$, la sortie est $\Omega_m(p)$.}}
\ifprof
\begin{corrige}


\footnotesize
%\begin{figure}[!h]
\begin{center}

\begin{tikzpicture}
\sbEntree{E}
%
%\sbBloc[5]{b0}{Adapt.}{E}
%    \sbRelier[$\omega_{\text{cons}}(t)$]{E}{b0}

\sbComp{c1}{E}
    \sbRelier[$U_m(p)$]{E}{c1}

\sbBloc{b1}{$\dfrac{1}{R+Lp}$}{c1}
    \sbRelier{c1}{b1}
    
\sbBloc[4]{b2}{$k_c$}{b1}
    \sbRelier[$I(p)$]{b1}{b2}


\sbSumh[6]{c2}{b2}
    \sbRelier[$C_m(p)$]{b2}{c2}

%\draw[latex−](c2)|−++(−3em,3em)node[left]{$C_r(p)$};
%\draw(c2)(−3em,3em)node[left]{$C_r(p)$};

\draw [latex-](c2)|- ++ (-3em,3em) node[left]{$C_r(p)$};

\sbBloc[3]{b3}{$\dfrac{1}{Jp}$}{c2}
    \sbRelier{c2}{b3}

\sbSortie[4]{S}{b3}
    \sbRelier{b3}{S}
    \sbNomLien[0.8]{S}{$\Omega_m(p)$}
  
%\sbRenvoi{b2-S}{c1}{}


%% BLOC DE RETOUR ET RENVOI
\sbDecaleNoeudy[4]{b3}{n1} 
\sbDecaleNoeudx[0]{n1}{n2} 

\sbBlocr{r}{$k_e$}{n2} 
\sbRelieryx{b3-S}{r}
\sbRelierxy[$E(p)$]{r}{c1}
%%%%

\end{tikzpicture}
\end{center}


Le moteur à courant continu seul ne possède pas de capteur. Il n'y a donc pas de grandeur asservie
\end{corrige}
\else
\fi

\begin{question}\label{cy_13_ch_03_td_01_q06}
En supposant le couple résistant nul, $c_r(t) = 0$, donner la forme canonique de la fonction de
transfert  sous la forme $H_m(p) = \ordredeux$. On exprimera les constantes en fonction de $R$, $L$, $k_e$, $k_c$ et $J$.
\end{question}
\ifprof
\begin{corrige}
On utilise la formule de Black et on a :
$H_m(p)=\dfrac{\dfrac{1}{R+Lp}\dfrac{k_c}{Jp}}{1+k_e\dfrac{1}{R+Lp}\dfrac{k_c}{Jp}}$
$=\dfrac{k_c}{\left(R+Lp\right)Jp+k_ck_e}$

En mettant sous forme canoniquen, on a alors $H_m(p)=\dfrac{\dfrac{1}{k_e}}{\dfrac{JL}{k_ck_e}p^2 +\dfrac{JR}{k_ck_e}p + 1}$.

Par identification, on a donc $K= \dfrac{1}{k_e}$, $\omega_0 =\sqrt{ \dfrac{k_ck_e}{JL}}$ et
$\dfrac{2\xi}{\omega_0}= \dfrac{JR}{k_ck_e}$ soit $\xi =  \dfrac{JR}{2 k_ck_e}\sqrt{ \dfrac{k_ck_e}{JL}} = $
$=  \dfrac{R\sqrt{J}}{2 \sqrt{L k_ck_e}} $.

\end{corrige}
\else
\fi


\subsection*{Optimisation des performances de l’asservissement en vitesse du moteur}
\begin{obj}
Analyser les performances de l’asservissement en vitesse du moteur afin de concevoir un
correcteur permettant de vérifier l’exigence 1.2.2.1.
\end{obj}


\ifprof
\else
\fi

Le correcteur de l’asservissement en vitesse du moteur est un proportionnel-intégrateur de fonction
de transfert $\indice{H}{cor}(p) = K_p+\dfrac{K_i}{p}$.

Toutes les fonction de transfert introduires dans la question \ref{cy_13_ch_03_td_01_q01} sont maintenant connues. On conserve $H_m(p)$ sous la forme introduite question \ref{cy_13_ch_03_td_01_q06}.
%La fonction de transfert en boucle fermée est donnée par 
%$\indice{H}{BF}(p) = \dfrac{\Omega_m(p)}{\indice{\Omega}{cons}(p)} =  \dfrac{K_c\indice{H}{cor}(p) K}{1+\dfrac{2\xi}{\omega_0}p+\dfrac{p^2}{\omega_0^2}+ \indice{H}{cor}(p) K_c}$

\question{Déterminer la fonction de transfert en boucle fermée notée $\indice{H}{BF}(p) = \dfrac{\Omega_m(p)}{\indice{\Omega}{cons}(p)}$.}
\ifprof
\begin{corrige}
On a $\indice{H}{cor}(p) =\dfrac{K_p p + K_i }{p}$.

En utilisant la formule de Black, on a 
$\indice{H}{BF}(p) = K_a \dfrac{\indice{H}{cor}(p) H_m(p)}{1+\indice{H}{cor}(p) H_m(p)K_a}$
$= K_a \dfrac{\dfrac{K_p p + K_i }{p} \ordredeux}{1+\dfrac{K_p p + K_i }{p} \ordredeux K_a}$
$=\dfrac{ K_a K \left(K_p p + K_i\right)}{p\left( 1+\dfrac{2\xi}{\omega_0}p+\dfrac{p^2}{\omega_0^2}\right)+\left(K_p p + K_i \right)K K_a}$
$=\dfrac{ K_a K \left(K_p p + K_i\right)}{p\left( 1+\dfrac{2\xi}{\omega_0}p+\dfrac{p^2}{\omega_0^2}\right)+K_p K K_a p + K_iK K_a}$
$=\dfrac{ K_a K \left(K_p p + K_i\right)}{p\left( 1+\dfrac{2\xi}{\omega_0}p+\dfrac{p^2}{\omega_0^2}\right)+K_p K K_a p + K_iK K_a}$

\end{corrige}
\else
\fi

\question{Mettre $\indice{H}{BF}(p)$ sous forme canonique. Donner son gain statique, son ordre et sa classe.}
\ifprof
\begin{corrige}
$\indice{H}{BF}(p)=\dfrac{  \left(\dfrac{K_p}{K_i} p + 1\right)}{\dfrac{p}{K_iK K_a}\left( 1+\dfrac{2\xi}{\omega_0}p+\dfrac{p^2}{\omega_0^2}\right)+\dfrac{K_p}{K_i} p + 1}$. Le gain est unitaire, l'ordre est 3, la classe est nulle.

\end{corrige}
\else
\fi

\question{Déterminer la valeur finale de $\omega_m(t)$ pour une entrée indicielle. Conclure vis-à-vis du cahier des charges}
\ifprof
\begin{corrige}
On a $\lim\limits_{t\to  +\infty}  \omega_m(t)= \lim\limits_{p\to  0}  p \omega_m(p)  $
$= \lim\limits_{p\to  0}  p \indice{\omega}{cons}(p)  \indice{H}{BF}(p) $. 

Pour une entrée indicielle, 
$\indice{\omega}{cons}(p) = \dfrac{1}{p}$ soit 
$= \lim\limits_{p\to  0}    \indice{H}{BF}(p) $
$= 1$.

Pour une entrée indicielle, la sortie vaut également 1. L'erreur est donc nulle. La précision est donc absolue, comme demandé dans l'exigence 1.2.2.1.
\end{corrige}
\else
\fi

\question{Déterminer la valeur finale de $\omega_m(t)$ pour une entrée rampe de pente 1.}
\ifprof
\begin{corrige}
\end{corrige}
\else
\fi



\subsection*{Evaluation graphique des performances}

Les résultats de simulation de la réponse du moteur $\indice{N}{m}(t)$, en boucle fermée, pour une entrée échelon
d’amplitude $N_0 = \SI{3 000}{tr.min^{-1}}$ pour différentes valeurs de $K_p$ et de $K_i$ sont donnés sur la figure \ref{cy_01_ch_02_03_td_03_fig_05}.



\ifprof
\begin{marginfigure}
\includegraphics[width=\linewidth]{fig_05}
\caption{Évolutions simulées de $\omega_m(t)$.
\label{cy_01_ch_02_03_td_03_fig_05}}
\end{marginfigure}

\else
\begin{figure}[!h]
\includegraphics[width=\linewidth]{fig_05}
\caption{Évolutions simulées de $\omega_m(t)$.
\label{cy_01_ch_02_03_td_03_fig_05}}
\end{figure}
\fi

%
%
%\begin{question}
%Pour les courbes 1 et 2 de la figure \ref{cy_01_ch_02_03_td_03_fig_05}, préciser, en le justifiant, la simulation qui est associée à la plus grande
%valeur de $K_p$. On pourra exprimer le coefficient d'amortissement de la FTBF ou exprimer l'écart statique.
%\end{question}
%\ifprof
%\begin{corrige}
%\textbf{Méthode 1 -- Coefficient d'amortissement}
%
%On note $\indice{H}{BF}(p) = \dfrac{\Omega_m(p)}{\indice{\Omega}{cons}(p)}$.
%
%
%On a alors, $\indice{H}{BF}(p) = K_c \dfrac{K_p \ordredeux }{1+K_p \ordredeux  K_c}$
%$=  \dfrac{K_cK_p K}{1+\dfrac{2\xi}{\omega_0}p+\dfrac{p^2}{\omega_0^2}+K_p K_c}$.
%
%On a donc $\dfrac{2\indice{\xi}{BF}}{\indice{\omega}{BF}} = \dfrac{2\xi}{\omega_0\left(1+K_p K_c\right)}$
%et $\indice{\omega}{BF}^2 = \omega_0^2 \left(1+K_p K_c\right)$.
%
%Soit
%$\indice{\xi}{BF} = \dfrac{\xi\indice{\omega}{BF}}{\omega_0\left(1+K_p K_c\right)}$ 
%$= \dfrac{\xi\omega_0\sqrt{1+K_p K_c}}{\omega_0\left(1+K_p K_c\right)}$ 
%$= \dfrac{\xi}{\omega_0\sqrt{1+K_p K_c}}$.
%
%En conclusion, plus $K_p$ augmente, plus le coefficient d'amortissement diminue et donc plus les pseudo oscillations deviennent grandes. La courbe 2 a donc la plus grande valeur de $K_p$.
%
%\textbf{Méthode 2 -- Calcul de l'écart statique}
%
%On montre que $\varepsilon(p) = \indice{\omega}{cons}(p) K_a \dfrac{1}{1+FTBO(p)}$
%$= \dfrac{\indice{\omega}{cons}(p) K_a}{1+K_p K_c\ordredeux }$.
%
%Pour une entrée échelon et en utilisant le théorème de la valeur finale, on a 
%$\varepsilon_S =  \lim\limits_{t\to +\infty} = \varepsilon(t) = \lim\limits_{p\to 0} p \varepsilon(p)$
%$\lim\limits_{p\to 0} \dfrac{ K_a}{1+K_p K_c\ordredeux } = \dfrac{K_a}{1+K_p K_c K}$.
%
%Lorsque $K_p$ augmente, $ \varepsilon_S$ diminue. La courbe 2 a donc la plus grande valeur de $K_p$.
%
%
%
%\end{corrige}
%\else
%\fi
%
%\begin{question}
%Pour chaque courbe de la figure \ref{cy_01_ch_02_03_td_03_fig_05}, préciser, en le justifiant, si la valeur de $K_i$ est nulle ou non.
%\end{question}
%\ifprof
%\begin{corrige}
%On montre que $\varepsilon(p) = \indice{\omega}{cons}(p) K_a \dfrac{1}{1+FTBO(p)}$
%$= \dfrac{\indice{\omega}{cons}(p) K_a}{1+\left( K_p + \dfrac{K_i}{p}\right) K_c\ordredeux }$.
%
%Pour une entrée échelon et en utilisant le théorème de la valeur finale, on a 
%$\varepsilon_S =  \lim\limits_{t\to +\infty} = \varepsilon(t) = \lim\limits_{p\to 0} p \varepsilon(p) = 0$.
%Ainsi, si $K_i$ non nul, $\varepsilon_S  =0$ (courbe 3 uniquement).
%
%\end{corrige}
%\else
%\fi
%
%


\begin{question}
Déterminer les valeurs associées aux quatre critères de performances de l’exigence 1.2.2.1.
Conclure sur le correcteur à adopter.
\end{question}
\ifprof
\begin{corrige}
\footnotesize
\begin{tabular}{llllll}
\hline 
 & Stabilité & 1\ier Dépassement & Erreur statique & $T_{5\%}$ \\
 \hline 
 Exigences &  Absolue & $< 20\, \%$ & Nulle & $\SI{0,5}{s}$ \\
 Courbe 1 & Stable \textbf{OK} & $D_1 = \SI{45}{\%}$ \textbf{Pas OK} &$\SI{2450}{tr/min} $ \textbf{Pas OK}&$T_{5\%} = \SI{0,015}{s}$ \textbf{ OK}\\
 Courbe 2 & Stable \textbf{OK} & $D_1 = \SI{59}{\%}$ \textbf{Pas OK} &$\SI{900}{tr/min} $ \textbf{Pas OK}&$T_{5\%} = \SI{0,018}{s}$ \textbf{ OK}\\
  Courbe 3 & Stable \textbf{OK} & $D_1 = \SI{15}{\%}$ \textbf{OK} &$\SI{0}{tr/min} $ \textbf{OK}&$T_{5\%} = \SI{0,048}{s}$ \textbf{ OK}\\
 \hline
 \end{tabular}

\normalsize
\end{corrige}
\else
\fi



\ifprof
\else
\begin{marginfigure}
\centering
%\includegraphics[width=3cm]{Cy_01_Ch_02_03_TD_04_qr}
\end{marginfigure}
\fi

\ifcolle
\else
\ifprof
\else
%\marginnote{
\begin{solution}
\begin{enumerate}[wide, labelwidth=!, labelindent=0pt]
\item .
\item $K_a = K_c$.
\item .
\item $K= \dfrac{1}{k_e}$, $\omega_0 =\sqrt{ \dfrac{k_ck_e}{JL}}$ et
$\xi =  \dfrac{R\sqrt{J}}{2 \sqrt{L k_ck_e}} $.
\item La courbe 2 a la plus grande valeur de $K_p$.
\item $K_i\neq 0 $ pour la courbe 3 uniquemement.
\item .
\end{enumerate}
\end{solution}
%}
\fi
\fi




