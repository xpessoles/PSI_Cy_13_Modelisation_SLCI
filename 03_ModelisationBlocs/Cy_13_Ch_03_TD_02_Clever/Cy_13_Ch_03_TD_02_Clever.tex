% "{'chapitre':'slci_laplace','classe':('PSI'),'type':('td'),'titre':'Vehicule Clever à 3 roues', 'source':'Concours Banque PT SIA 2013','comp':['SLCI-02','SLCI-07','SLCI-08'],'corrige':False}"

% Comp ci dessus à revoir
\setchapterimage{fig_00}


\renewcommand{\titrechapitre}{Véhicule à trois roues Clever}

\chapter*{TD \arabic{cptTD} \\ 
\titrechapitre\,-- \ifprof Corrigé \else Sujet \fi}
\addcontentsline{toc}{section}{TD \arabic{cptTD} : \titrechapitre -- \ifprof Corrigé \else Sujet \fi}

\iflivret \stepcounter{cptTD} \else
\ifprof  \stepcounter{cptTD} \else \fi \fi

\renewcommand{\leftmark}{\titrechapitre}
\renewcommand{\rightmark}{\titrechapitre}


\setcounter{question}{0}

\marginnote{\textit{Concours Banque PT SIA 2013.}}
%\marginnote{\UPSTIcompetence[2]{B2-07}}
\marginnote{\xpComp{SLCI}{02}\xpComp{SLCI}{07}\xpComp{SLCI}{08} -- À vérifier}



\begin{marginfigure}
\centering
\includegraphics[width=.9\linewidth]{fig_01}
\includegraphics[width=.9\linewidth]{fig_02}
\end{marginfigure}

\ifprof
\else
Le Clever est un démonstrateur technologique développé par un tissu d'industriels européens -- dont BMW, l'Institut Français du Pétrole (IFP) et de nombreux équipementiers --. Clever est la contraction de Compact Low Emission VEhiclefor uRban tRansportation (véhicule compact à faibles émissions pour le transport urbain). 


Du point de vue de l'architecture cinématique, le groupe motopropulseur est placé à l'arrière. À l'avant, l'habitacle repose sur une roue de moto et pivote par rapport au bloc arrière autour d'une liaison pilotée angulairement par le biais de deux vérins hydrauliques. L'inclinaison est contrôlée par un ordinateur de bord en fonction de l'angle au volant et de la vitesse. 

Le cahier des charges de l'asservissement angulaire de l'inclinaison de l'habitacle est donné par la table \ref{cy_13_ch_01_td_02_tab_1}.

\begin{center}%[h]
%\centering
\begin{tabular}{p{2.5cm}p{6cm}l}
\hline
\textbf{Exigence} & \textbf{Critères} & \textbf{Niveau} \\
\hline 
\multirow{4}{2.5cm}{Contrôler le mouvement de l'habitacle} 
& Ecart statique pour une entrée indicielle & $0\degres$ \\
& Ecart de trainage pour une entrée en rampe unitaire & $0\degres$ \\
& Ecart dynamique & $<1\degres$ \\
& Temps de réponse à 5\% & $\leq \SI{0,1}{s}$ \\
%& Marge de phase & Comprise entre $45\degres$ et $50\degres$ \\
& Dépassement & Inférieur à 10\% \\
\hline
\end{tabular}
\captionof{table}{Exigences de l'asservissement \label{cy_13_ch_01_td_02_tab_1}}
\end{center}
\fi



\subsection*{Modélisation de la structure du système}

\begin{marginfigure}
\centering
\includegraphics[width=.9\linewidth]{fig_03}
\end{marginfigure}

\ifprof
\else
Le mouvement de l'habitacle est asservi en position angulaire. La consigne $\alpha_c(t)$ est générée par un calculateur puis adaptée par un \textbf{adaptateur} électronique, dont le signal de sorti est noté $u_c(t)$. Ce signal est \textbf{comparé} au signal $m(t)$ provenant d'un \textbf{capteur de position}. En sortie du comparateur, l'écart $\varepsilon(t)$ est ajusté grâce à un \textbf{correcteur}. 
Le correcteur, grâce à une tension de commande $u(t)$, pilote un \textbf{servo-distributeur hydraulique} dont l'objectif est de moduler le débit d'huile $q(t)$ qui permet de déplacer le \textbf{vérin}. Ce déplacement est noté $\lambda(t)$. Ce vérin permet de modifier les longueurs du \textbf{triangle} $OA_1B_1$ et donc de modifier l'angle $\alpha(t)$ de  l'habitacle.
\fi

\question{Proposer un schéma-bloc fonctionnel pour modéliser l'asservissement en position angulaire de l'habitacle.}
\ifprof
\begin{corrige}
\footnotesize
%\begin{figure}[!h]
\begin{center}
\begin{tikzpicture}
\sbEntree{E}

\sbBloc[5]{b0}{Adapt.}{E}
    \sbRelier[$\alpha_{\text{c}}(t)$]{E}{b0}

\sbComp{c1}{b0}
    \sbRelier{b0}{c1}

\sbBloc[3]{b1}{Corr.}{c1}
    \sbRelier[$\varepsilon(t)$]{c1}{b1}
    
\sbBloc[3]{b2}{Servo dist. }{b1}
    \sbRelier[$u(t)$]{b1}{b2}

\sbBloc[3]{b3}{Vérin}{b2}
    \sbRelier[$q(t)$]{b2}{b3}


\sbBloc[3]{b4}{Triangle}{b3}
    \sbRelier[$\lambda(t)$]{b3}{b4}


\sbSortie[4]{S}{b4}
    \sbRelier{b4}{S}
    \sbNomLien[0.8]{S}{$\alpha(t)$}
  
%\sbRenvoi{b2-S}{c1}{}


%% BLOC DE RETOUR ET RENVOI
\sbDecaleNoeudy[4]{b4}{n1} 
\sbDecaleNoeudx[0]{n1}{n2} 

\sbBlocr{r}{Capteur}{n2} 
\sbRelieryx{b4-S}{r}
\sbRelierxy[$m(t)$]{r}{c1}
%%%%

\end{tikzpicture}
\end{center}
\normalsize
%%%%%%%%%%%%%%%
\end{corrige}
\else
%%%%%%%%%%%%%%%%%%%%%
\footnotesize
\begin{figure}[!h]
\begin{center}
\begin{tikzpicture}
\sbEntree{E}

\sbBloc[5]{b0}{ }{E}
    \sbRelier[$\omega_{\text{cons}}(t)$]{E}{b0}

\sbComp{c1}{b0}
    \sbRelier{b0}{c1}

\sbBloc[3]{b1}{}{c1}
    \sbRelier{c1}{b1}
    
\sbBloc[3]{b2}{ }{b1}
    \sbRelier[$ $]{b1}{b2}
    

\sbSortie[4]{S}{b2}
    \sbRelier{b2}{S}
    \sbNomLien[0.8]{S}{$\omega_m(t)$}
  
%\sbRenvoi{b2-S}{c1}{}


%% BLOC DE RETOUR ET RENVOI
\sbDecaleNoeudy[4]{b2}{n1} 
\sbDecaleNoeudx[0]{n1}{n2} 

\sbBlocr{r}{}{n2} 
\sbRelieryx{b2-S}{r}
\sbRelierxy{r}{c1}
%%%%

\end{tikzpicture}
\end{center}

\end{figure}
\normalsize
%%%%%%%%%%%%%%%
\fi



\question{Cet asservissement satisfait-il les hypothèses des SLCI (Systèmes Linéaires Continus Invariants) ?}
\ifprof
\begin{corrige}
\begin{itemize}
\item Le système est asservi car il est équipé d'un capteur et d'un sommateur qui permettent de corriger l'écart entre la consigne de l'utilisateur et le signal mesuré sur le système.
\item Linéarité du système : la transformation géométrique du triangle n'est pas linéaire.
\item Continuité du système : peu d'information ici, mais l'asservissement sera très certainement fait par un micocontroleur fonctionnant avec des grandeurs numériques, les CAN et les CNA peuvent remettre en question la continuité du système.
\item Invariance du système : on fait l'hypothèse que les constantes ne varient pas au cours du temps.
\end{itemize}
\end{corrige}
\else
\fi


\subsection*{Modèle de connaissance}
\ifprof
\else
Les équations suivantes modélisent le comportement des composants :
\begin{itemize}
\item comparateur : $\varepsilon(t)=u_c(t)-m(t)$, correcteur : $u(t)=K\varepsilon(t)$;
\item servo-distributeur : $q(t)=K_S u(t)$;
\item capteur : $m(t)=C\alpha(t)$, adapteur électronique : $u_c(t)=C\alpha_c(t)$;
\item en faisant l'hypothèse que le fluide est incompressible : $q(t)=S\lambdap(t)$;
\item en faisant l'hypothèse que les variation de $\alpha(t)$ sont limitées, on a $\alpha(t)=R\lambda(t)$.
%($\lambdap(t)=\dfrac{\d\lambda(t)}{\d t}$).
\end{itemize}

\marginnote{On notera $U_c(p) = \mathcal{L}\left(u_c(t)\right)$, 
$M(p) = \mathcal{L}\left(m(t)\right)$, 
$\varepsilon(p) = \mathcal{L}\left(\varepsilon(t)\right)$, 
$Q(p) = \mathcal{L}\left(q(t)\right)$, 
$U(p) = \mathcal{L}\left(u(t)\right)$, 
$\alpha(p) = \mathcal{L}\left(\alpha(t)\right)$,
$\lambda(p) = \mathcal{L}\left(\lambda(t)\right)$...}
\fi


\question{En faisant l'hypothèse que les conditions de Heaviside sont respectées, transformer les équations dans le domaine de Laplace.}
\ifprof
\begin{corrige}
\begin{itemize}
\item comparateur : $\varepsilon(p)=U_c(p)-M(p)$;
\item correcteur : $U(p)=K\varepsilon(p)$;
\item servo-distributeur : $Q(p)=K_S U(p)$;
\item capteur : $M(p)=C\alpha(p)$;
\item adapteur électronique : $U_c(t)=C\alpha_c(p)$;
\item en faisant l'hypothèse que le fluide est incompressible : $Q(p)=Sp \lambda(p)$;
\item  $\alpha(p)=R\lambda(p)$.
\end{itemize}
\end{corrige}
\else
\fi


\question{Donner le schéma-bloc du système.}

\question{Calculer la fonction de transfert $F(p) = \dfrac{M(p)}{\varepsilon(p)}$. Mettre cette fonction sous forme canonique. Préciser son ordre, son gain statique, sa classe et ses pôles.}
\ifprof
\begin{corrige}
En utilisant les équations précédentes, on a : 
$M(p)=C\alpha(p)$ $=CR\lambda(p)$ $= \dfrac{CR}{Sp} Q(p)$ $= \dfrac{CR}{Sp} K_S U(p)$
$= \dfrac{CR}{Sp} K_S K\varepsilon(p)$. 
On  a donc $F(p) = \dfrac{M(p)}{\varepsilon(p)} = \dfrac{CR}{Sp} K_S K$.

Il s'agit d'une fonction de transfert d'ordre 1 et de classe 1, de gain $\dfrac{CRK_S K}{S} $. Cette fonction a un pôle nul.
\end{corrige}
\else
\fi


\question{Calculer la fonction de transfert $H(p) = \dfrac{\alpha(p)}{\alpha_c(p)}$. Mettre cette fonction sous forme canonique. Préciser son ordre, son gain statique, sa classe et ses pôles.}
\ifprof
\begin{corrige}
On a $\varepsilon(p)=U_c(p)-M(p)$
$\Rightarrow \dfrac{M(p)}{F(p)}=U_c(p)-M(p)$
$\Rightarrow M(p) \left(\dfrac{1}{F(p)} + 1\right)=U_c(p)$
$\Rightarrow C\alpha(p) \left(\dfrac{1}{F(p)} + 1\right)=C \alpha_c(p)$
$\Rightarrow H(p) = \dfrac{\alpha(p)}{\alpha_c(p)} =  \dfrac{1}{\dfrac{1}{F(p)} + 1}$
$\Rightarrow H(p) =  \dfrac{F(p)}{1+F(p)}$
$\Rightarrow H(p) =  \dfrac{\dfrac{CR}{Sp} K_S K}{1+\dfrac{CR}{Sp} K_S K}$
$\Rightarrow H(p) =  \dfrac{CRK_S K}{Sp+CR K_S K}$
$\Rightarrow H(p) =   \dfrac{1}{\dfrac{S}{CRK_S K}p+1}$


C'est une fonction de transfert d'ordre 1, de classe 0, de gain statique 1 ayant un seul pôle : $-\dfrac{CRK_S K}{S}$.
\end{corrige}
\else
\fi

\subsection*{Evaluation analytique des performances}

\question{Donner la transformée de Laplace d'un échelon d'amplitude 1 et d'une rampe de pente 1.}
\ifprof
\begin{corrige}
$\alpha_c(p) = \dfrac{1}{p}$ pour un échelon et $\alpha_c(p) = \dfrac{1}{p^2}$ pour une rampe.
\end{corrige}
\else
\fi


\question{Déterminer la valeur finale de $\alpha_c(t)$ pour une entrée échelon. Conclure vis-à-vis du cahier des charges.}
\ifprof
\begin{corrige}
On a $\alpha(p)=\alpha_c(p) H(p)$. 
Par ailleurs, 
$\lim\limits_{t\to  +\infty}  \alpha(t) = \lim\limits_{t\to  0}  p  \alpha(p) $ 
$= \lim\limits_{t\to  0}  p  \dfrac{1}{p} \dfrac{1}{\dfrac{S}{CRK_S K}p+1} = 1$

Quand $t$ tend vers l'infini, la valeur finale est égale à la valeur d'entrée. Le système a un écart statique nul.

\end{corrige}
\else
\fi

\question{Déterminer la valeur finale de $\alpha_c(t)$ pour une entrée en rampe. Peut-on conclure sur l'exigence concernant l'écart dynamique ?}
\ifprof
\begin{corrige}
$\lim\limits_{t\to  +\infty}  \alpha(t) =  $ 
$= \lim\limits_{t\to  0}  p  \dfrac{1}{p^2} \dfrac{1}{\dfrac{S}{CRK_S K}p+1} $
$= \lim\limits_{t\to  0}   \dfrac{1}{p} \dfrac{1}{\dfrac{S}{CRK_S K}p+1}  = +\infty$

On ne peut pas conclure. Il faudrait calculer 
$\lim\limits_{t\to  +\infty}  \alpha_c(t) - \alpha(t)$.
\end{corrige}
\else
\fi

\begin{marginfigure}
\includegraphics[width=\linewidth]{fig_04}
\end{marginfigure}

\subsection*{Evaluation graphique des performances}
\ifprof
\else
Le correcteur a été modifié pour améliorer les performances du système.
On sollicite le système avec l'entrée ci-contre.
On mesure la réponse suivante.
\fi


\ifprof
\begin{marginfigure}
\centering
\includegraphics[width=\linewidth]{fig_05}
\end{marginfigure}
\else
\begin{center}
\centering
\includegraphics[width=.5\linewidth]{fig_05}
\end{center}
\fi

\question{Le système satisfait-il les exigences du cahier des charges ?}
\ifprof
\begin{corrige}
\begin{itemize}
\item Sur la partie entrée en rampe, entre 0,4 et 0,7 seconde, l'écart de trainage semble tendre vers 0\degres. (CDC OK)
\item Sur la partie finale, $t>\SI{1,4}{s}$, l'écart statique smeble tendre vers 0\degres. (CDC OK)
\item Le temps de réponse ne peut pas être vérifié sur cet essai. 
\item Le dépassement est de $3/45$ soit 7\%  ce qui est compatible avec le CDC.  
\end{itemize}
\end{corrige}
\else
\fi

